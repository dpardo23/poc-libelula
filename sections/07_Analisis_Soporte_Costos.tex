\documentclass[12pt,a4paper]{article}
\usepackage[utf8]{inputenc}
\usepackage[T1]{fontenc}
\usepackage[spanish]{babel}
\usepackage{geometry}
\usepackage{booktabs}
\usepackage{array}
\usepackage{graphicx}
\usepackage{hyperref}
\geometry{margin=2.5cm}

\title{\textbf{Análisis de Viabilidad y Soporte de Costos de la Pasarela de Pagos Libélula}}
\author{Equipo de Análisis Financiero y Tecnológico}
\date{\today}

\begin{document}
	
	\section*{Descripción general}
	\noindent
	Libélula es una pasarela de pagos diseñada para facilitar \textbf{transacciones directas entre clientes y proveedores}, sin funcionar como un tercero que retiene o distribuye fondos. Su arquitectura prioriza la \textbf{simplicidad operativa}, la \textbf{seguridad} y la \textbf{transparencia contable}, permitiendo que el dinero pase directamente del pagador al receptor, con registro y trazabilidad total.
	
	\section*{Modelo de operación}
	\noindent
	El modelo operativo de Libélula se basa en los siguientes principios:
	\begin{itemize}
		\item \textbf{Relación directa:} Cliente $\rightarrow$ Proveedor (sin intermediarios financieros).
		\item \textbf{Registro de pagos:} Cada transacción se almacena como un pago único, verificable y auditable.
		\item \textbf{Comisión por servicio:} Libélula aplica una comisión porcentual sobre cada transacción procesada, destinada a cubrir infraestructura, seguridad y soporte técnico.
	\end{itemize}
	
	Este enfoque evita los costos de custodia y distribución de fondos que manejan las pasarelas multilaterales (como Stripe o MercadoPago), lo que mejora la eficiencia operativa y reduce los riesgos financieros.

	\section*{Comparativa con otras pasarelas}
	
	\begin{center}
		\begin{tabular}{>{\raggedright}p{4cm} p{5cm} p{5cm}}
			\toprule
			\textbf{Característica} & \textbf{Libélula} & \textbf{Stripe / MercadoPago} \\
			\midrule
			Tipo de flujo & Directo (Cliente $\rightarrow$ Proveedor) & Intermediado (Cliente $\rightarrow$ Pasarela $\rightarrow$ Proveedor) \\
			Comisión promedio & 2.5\% & 3.5\% -- 5\% \\
			Tiempo de liquidación & Inmediato o 24 horas & 1--5 días hábiles \\
			Custodia de fondos & No aplica & Sí (retención temporal) \\
			Transparencia contable & Alta & Media \\
			Complejidad técnica & Baja (API simplificada) & Media / Alta \\
			Costo operativo interno & Bajo & Alto (múltiples actores) \\
			\bottomrule
		\end{tabular}
	\end{center}
	
	\section*{Evaluación de viabilidad}
	\subsection*{a) Económica}
	El modelo de Libélula es viable con una comisión del 2.5\% por transacción, siempre que el volumen mensual supere el umbral de sostenibilidad (por ejemplo, 2,000 transacciones mensuales). A menores volúmenes, el costo fijo tiene un mayor impacto, pero sigue siendo competitivo frente a las comisiones de los principales competidores.
	
	\subsection*{b) Operativa}
	El flujo directo simplifica la conciliación y reduce errores contables. La ausencia de custodia de fondos minimiza riesgos regulatorios y financieros, al tiempo que mejora la velocidad de liquidación.
	
	\subsection*{c) Técnica}
	La infraestructura requerida es escalable y puede implementarse sobre servicios en la nube (AWS, DigitalOcean, etc.), sin necesidad de módulos complejos de reparto de pagos. Esto reduce los costos de mantenimiento y facilita la adopción tecnológica.
	
	\subsection*{d) Competitiva}
	Libélula ofrece menores comisiones, tiempos de pago más rápidos y una integración más ligera. Su principal limitación ---la falta de soporte para pagos multiproveedor--- se compensa con una mayor eficiencia y menor costo operativo.
	\end{document}