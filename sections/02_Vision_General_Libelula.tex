\section{Visión General de Libélula como Plataforma de Pagos}

    \subsection{Descripción de la Empresa y Posicionamiento en el Mercado Boliviano}
        Libélula se define como una pasarela de pagos online orientada al cobro de cuentas. Es una solución tecnológica perteneciente 
        a la empresa boliviana \textbf{Todotix SRL}. Su modelo de negocio está enfocado en el segmento B2B (Business-to-Business), 
        proveyendo a las empresas una plataforma centralizada para que estas puedan gestionar y recibir pagos de sus clientes finales.\par

        El posicionamiento estratégico de Libélula en el mercado boliviano radica en su enfoque local, ofreciendo una integración 
        directa con los métodos de pago de mayor uso y confianza en el país, los cuales a menudo no están disponibles en pasarelas 
        internacionales.\par

        El paradigma técnico fundamental de la API de Libélula se basa en el concepto de \emph{registro de deuda} (\texttt{REGISTRAR DEUDA}). 
        A diferencia de otras pasarelas que crean intentos de pago inmediatos, el flujo estándar de Libélula consiste en:\par

        \begin{enumerate}
            \item La empresa (nuestro \emph{backend}) notifica a la API de Libélula sobre una nueva deuda pendiente mediante una llamada \texttt{POST} al servicio web \texttt{/rest/deuda/registrar}.
            \item En esta llamada se envían todos los detalles de la transacción, incluyendo parámetros críticos como \texttt{appkey}, \texttt{identificador\_deuda}, \texttt{email\_cliente} y \texttt{lineas\_detalle\_deuda}.
            \item Si el registro es exitoso, Libélula responde con un JSON que contiene el parámetro \texttt{url\_pasarela\_pagos}.
            \item La empresa debe entonces redirigir al cliente final a esta URL, donde se le presentarán las opciones de pago (QR, Tarjeta, etc.) para saldar dicha deuda.
            \item Una vez que el pago es confirmado por el canal (ej. el banco), Libélula notifica a la empresa invocando el servicio \texttt{PAGO EXITOSO}, que debe estar implementado en nuestro \emph{backend} (la URL especificada en el parámetro \texttt{callback\_url} durante el registro), anexando el \texttt{transaction\_id} para la conciliación.
        \end{enumerate}

    \subsection{Propuesta de Valor: La Pasarela Multicanal}
        La principal propuesta de valor de Libélula es su capacidad para actuar como un \emph{hub} o agregador de múltiples canales 
        de pago bolivianos. Según el manual de integración, la plataforma unifica el acceso a los siguientes servicios:\par

            \begin{itemize}
                \item \textbf{Tarjetas de Débito y Crédito:} Integración con Cybersource Visa de ATC.
                \item \textbf{SimpleQR:} El estándar de pagos QR interoperable de Bolivia (en este caso, operado vía BCP).
                \item \textbf{Botón de Pagos:} Integración con la plataforma web del BCP.
                \item \textbf{Billetera Móvil:} Pagos a través de Tigo Money.
                \item \textbf{Banca por Internet:} Integración directa con la plataforma BNBNet del Banco Nacional de Bolivia.
            \end{itemize}

        Una ventaja estratégica clave, mencionada en la documentación, es que esta arquitectura permite a Libélula acoplar nuevos canales 
        de pago en el futuro de forma automática, sin implicar cambios o esfuerzos de desarrollo adicionales por parte de las empresas 
        ya integradas.\par

    \subsection{Identificación de Clientes y Casos de Uso Típicos}
        La plataforma está diseñada para cualquier empresa en Bolivia que necesite un mecanismo digital para el cobro de cuentas. 
        Los casos de uso técnicos se derivan directamente de los tipos de canales que ofrece:\par

        \begin{enumerate}
            \item \textbf{Caso de Uso 1: Pago Inmediato (E-commerce):} El cliente es redirigido a la \texttt{url\_pasarela\_pagos} y 
            paga en ese instante (ej. con tarjeta de crédito). Tras el pago, Libélula lo redirige de vuelta al sitio de la empresa 
            (usando el parámetro \texttt{url\_retorno}) y, simultáneamente, envía la confirmación al \emph{backend} (\texttt{callback\_url}).
            \item \textbf{Caso de Uso 2: Pago Diferido (Asíncrono):} Este es el escenario más relevante para SimpleQR o pagos bancarios. 
            El cliente genera la deuda, obtiene los datos (ej. ve el QR desde la \texttt{url\_pasarela\_pagos} o la \texttt{qr\_simple\_url}), 
            pero puede cerrar el navegador y realizar el pago horas después desde su aplicación bancaria. La documentación contempla este 
            escenario bajo el servicio \texttt{PAGO EXITOSO EN BACKGROUND}, indicando que la confirmación (el \emph{webhook}) se ejecutará 
            en segundo plano en el momento en que el banco notifique a Libélula, lo cual puede no ser inmediato.
        \end{enumerate}