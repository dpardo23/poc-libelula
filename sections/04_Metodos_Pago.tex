\documentclass[12pt,a4paper]{article}
\usepackage[utf8]{inputenc}
\usepackage[T1]{fontenc}
\usepackage[spanish]{babel}
\usepackage{geometry}
\usepackage{booktabs}
\usepackage{array}
\usepackage{graphicx}
\usepackage{hyperref}
\geometry{margin=2.5cm}

\begin{document}
		
\section*{Métodos de Pago de Libélula}

Libélula ofrece múltiples métodos de pago para facilitar transacciones directas entre clientes y proveedores, priorizando simplicidad, rapidez y seguridad. Los métodos disponibles son los siguientes:

\subsection*{1. Tigo Money}
\begin{itemize}
	\item \textbf{Descripción:} Servicio de billetera móvil ampliamente usado en Bolivia.
	\item \textbf{Flujo:} El cliente transfiere el monto desde su cuenta Tigo Money directamente al proveedor registrado en Libélula.
	\item \textbf{Ventajas:} Pago inmediato, sin necesidad de tarjeta bancaria, seguro y confiable.
\end{itemize}

\subsection*{2. Lula}
\begin{itemize}
	\item \textbf{Descripción:} Plataforma de pago digital que permite transacciones rápidas entre cuentas electrónicas.
	\item \textbf{Flujo:} El cliente autoriza el pago desde su cuenta Lula hacia el proveedor.
	\item \textbf{Ventajas:} Pago casi instantáneo, integración directa y sin intermediarios.
\end{itemize}

\subsection*{3. BNB (Banco Nacional de Bolivia)}
\begin{itemize}
	\item \textbf{Descripción:} Transferencias bancarias desde cuentas del BNB.
	\item \textbf{Flujo:} El cliente realiza la transferencia desde la plataforma del banco, acreditando el monto en la cuenta del proveedor a través de Libélula.
	\item \textbf{Ventajas:} Permite pagos desde cuentas bancarias tradicionales, con respaldo institucional.
\end{itemize}

\subsection*{4. Tarjeta de crédito o débito}
\begin{itemize}
	\item \textbf{Descripción:} Pagos con tarjetas Visa, Mastercard u otras aceptadas por la pasarela.
	\item \textbf{Flujo:} El cliente ingresa los datos de la tarjeta en la interfaz de Libélula; el sistema procesa el pago y acredita el monto al proveedor.
	\item \textbf{Ventajas:} Amplia accesibilidad y pagos rápidos sin necesidad de efectivo ni billeteras móviles.
\end{itemize}

\subsection*{5. QR simple}
\begin{itemize}
	\item \textbf{Descripción:} Pago mediante escaneo de código QR generado por Libélula.
	\item \textbf{Flujo:} El proveedor muestra un QR al cliente; este lo escanea con su app de billetera o banca móvil y confirma el pago.
	\item \textbf{Ventajas:} Rápido, no requiere introducir montos manualmente y reduce errores; ideal para pagos presenciales o a distancia.
\end{itemize}

\subsection*{Resumen Comparativo}
\begin{center}
	\begin{tabular}{>{\raggedright}p{3.5cm} p{3cm} p{3cm} p{5cm}}
		\toprule
		\textbf{Método} & \textbf{Tipo de pago} & \textbf{Rapidez} & \textbf{Ventaja principal} \\
		\midrule
		Tigo Money & Billetera móvil & Instantáneo & Sin necesidad de tarjeta, ampliamente usado \\
		Lula & Billetera digital & Instantáneo & Pago directo entre cuentas electrónicas \\
		BNB & Banco tradicional & Variable (1--2 días) & Seguridad y respaldo bancario \\
		Tarjeta & Crédito/débito & Instantáneo & Accesible y conveniente \\
		QR simple & Código QR & Instantáneo & Rápido, fácil y seguro para pagos presenciales \\
		\bottomrule
	\end{tabular}
\end{center}

	
\end{document}